\documentclass[a4page,11pt]{article}
\usepackage[utf8]{inputenc}
\usepackage{polski}
\textwidth=164mm \textheight=245mm
\advance\hoffset-.8in \advance\voffset-.7in
\parindent0pt \parskip.3\baselineskip
\renewcommand{\familydefault}{\sfdefault}
\raggedright
\title{Elementy statystyki}
\author{Tomasz Przechlewski}
\date{3/2019}
\begin{document}
\maketitle
%%

\section{Zadanie do wykonania}

Na podstawie danych z~szeregu szczegółowego: 1.~Wyznaczyć
szereg rozdzielczy (funkcja \texttt{CZĘSTOŚĆ/FREQUENCY}). 2.~wykreślić
histogram (na podstawie wartości szeregu rozdzielczego).
3.~Scharakteryzować średni poziom cechy, zróżnicowanie i~asymetrię
(szczegóły niżej)

\section{Szereg rozdzielczy/histogram}
Aby wykreślić histogram należy dane pogrupować w szereg
rozdzielczy. Do tego celu służy funkcja \texttt{CZĘSTOŚĆ}, której
składnia jest następująca:

\begin{verbatim}
CZĘSTOŚĆ (obszar-danych ; obszar-końców-przedziałów)
\end{verbatim}

W omawianym przykładzie obszar-danych, to kolumna zawierająca liczbę
hoteli w powiecie.

Argument obszar-końców-przedziałów to obszar górnych końców
przedziałów szeregu rozdzielczego. Funkcja \texttt{CZĘSTOŚĆ} zwraca
obszar o~jedną komórkę większy od obszaru górnych końców przedziałów
-- liczebność tej komórki, to liczba elementów większych od wartości
ostatniego górnego końca przedziału.


Uwaga: funkcja \texttt{CZĘSTOŚĆ} jest specjalna (bo zwraca obszar): po
jej wpisaniu do komórki należy nacisnąć \texttt{Ctrl-Shift-Enter} a
nie zwyczajne \texttt{Enter}.

Kształt histogramu (a co za tym idzie wnioski dotyczące kształtu
rozkładu) zależą w dużym stopniu od sposobu pogrupowania danych w
szereg rozdzielczy.

\section{Średni poziom cechy, zróżnicowanie i~asymetria}

Do obliczenia średniej, mediany, kwartyli, wariancji, odchylenia
standardowego itp. służą odpowiednie funkcje (podano
nazwy polskie/angielskie z~programu \texttt{OpenOffice~calc}):

\begin{verbatim}
SUMA (obszar-danych)        SUM(obszar-danych)
ŚREDNIA (obszar-danych)     AVERAGE(obszar-danych)
MEDIANA (obszar-danych)     MEDIAN(obszar-danych)
DOMINANTA (obszar-danych)   MODE(obszar-danych)
WARIANCJA (obszar-danych)   VAR(obszar-danych)
ODCHYL.STD(obszar-danych)   STDEV(obszar-danych)
SKOŚNOŚĆ (obszar-danych)    SKEW(obszar-danych)
KWARTYL(obszar-danych;typ)  QUARTILE(obszar-danych;typ)
jeżeli typ=0 KWARTYL oblicza wartość minimalną, jeżeli typ=4 maksymalną
\end{verbatim}

\section{Przykład}

[0] Plik \texttt{hotele.csv} zawiera liczbę całorocznych obiektów
hotelowych w~Polsce w~latach 2012 oraz 2017.  Źródłem danych jest
\texttt{https://bdl.stat.gov.pl/} (kategoria/grupa/podgrupa
Turystyka/Turystyczne obiekty noclegowe i~ich
wykorzystanie/Turystyczne obiekty noclegowe wg rodzajów). Celem
analizy jest porównanie struktury obu zbiorowości.

[1] Konstruujemy szereg rozdzielczy/histogram.

[2] Łącznie w~Polsce w badanych okresach było odpowiednio 3918 oraz
3326 hoteli całorocznych. W~roku 2012 średnio w powiecie było 8,9
hoteli całorocznych, podczas gdy w~roku 2017 takich hoteli było 10,3.
W~ciągu 5 lat nastąpił zatem wzrost o~1,4 hoteli (albo o
1,4/8,9=15,7\%)

[3] W~roku 2012 w~połowie powiatów było 5 hoteli i mniej,
a~połowie 5 hoteli i~więcej.

[4] W~roku 2012 w~połowie powiatów było 5 hoteli i~mniej, a~połowie 5
hoteli i~więcej. W~roku 2017 w~połowie powiatów było 7 hoteli i~mniej,
a~połowie 7~hoteli i~więcej.

[5] W~roku 2012 w~25\% powiatów było 3 hoteli i~mniej, a~75\% 3 hoteli
i~więcej.  W~roku 2017 w~25\% powiatów było 4 hoteli i~mniej, a~75\% 4
hoteli i~więcej.

[6] W~roku 2012 w~75\% powiatów było 10 hoteli i~mniej, a~25\% 10 hoteli
i~więcej.  W~roku 2017 w~75\% powiatów było 11 hoteli i~mniej, a~25\%
11~hoteli i~więcej.

[7] Jeżeli chodzi o~zróżnicowanie liczby hoteli to przeciętne odchylenie
od średniej arytmetycznej
wyniosło 13,5 hoteli (2012) oraz 15,7 hoteli (2017).
W~wartościach bezwzględnych odnotowano zatem niewielki wzrost zmienności.

[8] Zmienność względna (mierzona jako udział odchylenia standardowego
w średniej) spadła nieznacznie i~wyniosła odpowiednio 151,8\% oraz
154,2\%. Analiza zmienności wskazuje,
że przyrost liczby hoteli był równomierny.

[9] Wartość odchylenia ćwiartkowego była identyczna
w~obu badanych  okresach i~wyniosła 3,5 hoteli.

[10] Wartości klasycznego współczynnika asymetrii wyniosły odpowiednio
6,0 oraz 5,9 co wskazuje na znaczącą asymetrię
prawostronną. Współczynniki Pearsona z kolei mają wartości znacznie
niższe, bo wynoszą one 0,12 oraz 0,29 (odpowiednio).  Szczegółowa
analiza rozkład liczebności klas w~szeregu rozdzielczym (na terenie
75\% powiatów jest 0--6 hoteli) wskazuje że rozkład hoteli cechuje się
znaczącą skośnością dodatnią (prawostronną) w~obu analizowanych
okresach.  Niska wartość współczynnika Persona w omawianym przypadku
wskazuje, że dla rozkładu hoteli w~Polsce ten współczynnik nie jest
dobrą miarą oceny asymetrii.

\begin{small}
Objaśnienie:

[0] Populacja, cechy stałe. Cel analizy

[2] Średnia (suma jeżeli ma to sens)

[3] Mediana

[4] Dominanta (jeżeli ma sens -- w przypadku cechy ciągłej może nie mieć)

[5,6] Kwartyl1, Kwartyl3

[7] Odchylenie standardowe (wariancji się nie interpretuje)

[8] Współczynnik zmienności

[9] Odchylenie ćwiartkowe

[10] Skośność (współczynnik klasyczny; funkcja SKOŚNOŚĆ/SKEW)

\end{small}
\end{document}
